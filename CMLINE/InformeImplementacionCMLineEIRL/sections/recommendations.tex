\section{Recomendaciones}

\begin{itemize}
    \item \textbf{Mantener la disciplina en el ingreso de mercadería:} Es vital que toda reposición se registre mediante el módulo de Compras o Ajustes de Inventario al momento de la recepción física para mantener la veracidad del stock en tiempo real.
    \item \textbf{Actualización constante del catálogo:} Se recomienda revisar periódicamente las listas de precios y costos, especialmente en las conversiones de unidades (\textbf{Ciento/Millar}), para asegurar márgenes de ganancia correctos.
    \item \textbf{Finalizar el Branding:} Completar la definición de la identidad visual (logo y papelería) para personalizar los formatos PDF de cotizaciones y facturas, reforzando la imagen corporativa ante los clientes.
    \item \textbf{Evaluar módulos futuros:} Una vez consolidada la operación logística, se sugiere evaluar la implementación de los módulos de \textbf{CRM Avanzado} y \textbf{Contabilidad completa} para automatizar el flujo financiero y el seguimiento de oportunidades comerciales a largo plazo.
    \item \textbf{Uso estricto de la Clave SOL:} Mantener actualizadas las credenciales de \textbf{SUNAT} y monitorear el estado de los comprobantes electrónicos (\textbf{CDR}) para evitar contingencias fiscales.
\end{itemize}
