\section{Antecedentes del proyecto}
\subsection{Situación previa}
Antes del inicio de la implementación, el diagnóstico situacional de CM LINE reveló deficiencias operativas centradas principalmente en la gestión logística y su desconexión con el área comercial. La empresa operaba bajo un esquema de "ventas a ciegas", donde los vendedores no tenían certeza de la existencia real de la mercadería al momento de cerrar un trato, lo que ocasionaba quiebres de stock y pérdida de oportunidades de venta.

El problema central aquí es la incapacidad de gestionar eficientemente múltiples almacenes físicos distantes entre sí (Arequipa, Moquegua, Puno) y la falta de trazabilidad en productos sensibles como químicos de limpieza.

\begin{table}[h]
\centering
\begin{tabular}{|p{8cm}|p{8cm}|}
\hline
\textbf{Problema Detectado (Diagnóstico)} & \textbf{Solución Tecnológica (Odoo)} \\ 
\hline

\textbf{Incertidumbre de Stock:} Dificultad para conocer la disponibilidad de productos en tiempo real al momento de la venta (“ventas a ciegas”). 
& 
\textbf{Módulo de Inventario Integrado:} Visualización de stock disponible en tiempo real con bloqueo automático de ventas sin existencias. 
\\ \hline

\textbf{Descontrol Multi-Sede:} Gestión manual o desconectada de los tres almacenes (Arequipa, Moquegua y Puno), dificultando los traslados internos. 
& 
\textbf{Gestión Multi-Almacén Nativa:} Configuración centralizada de múltiples sedes con flujo automatizado de reabastecimiento y transferencias internas. 
\\ \hline

\textbf{Procesos Manuales en Despacho:} Emisión de guías de remisión y facturas desconectadas de la orden de venta. 
& 
\textbf{Facturación y Guías Electrónicas:} Emisión automática de comprobantes y Guías de Remisión Remitente integradas con SUNAT desde la misma orden de venta. 
\\ \hline

\textbf{Falta de Trazabilidad:} Riesgo de vender productos vencidos o defectuosos (detergentes, pañitos) por falta de control de lotes. 
& 
\textbf{Trazabilidad por Lotes:} Gestión de números de lote y fechas de caducidad para productos perecederos o químicos. 
\\ \hline

\end{tabular}
\end{table}

\subsection{Contexto empresarial de la empresa}

CM LINE E.I.R.L. es una empresa comercializadora con presencia regional en el sur del Perú, especializada en la distribución de materiales de construcción (cementos, fierros), artículos de ferretería, productos de limpieza y Equipos de Protección Personal (EPP),
.
Su modelo de negocio es híbrido y exigente, combinando la venta corporativa (cotizaciones formales, volúmenes altos, crédito) con la venta de mostrador (estilo retail, rápida y directa)

\subsection{Problemática identificada}