\section{Antecedentes del proyecto}

\subsection{Situación previa}
Antes del inicio de la implementación, el diagnóstico situacional de \textbf{CM LINE} reveló deficiencias operativas centradas principalmente en la gestión logística y su desconexión con el área comercial. La empresa operaba bajo un esquema de \textbf{"ventas a ciegas"}, donde los vendedores no tenían certeza de la existencia real de la mercadería al momento de cerrar un trato, lo que ocasionaba \textbf{quiebres de stock} y pérdida de oportunidades de venta.

El problema central era la incapacidad de gestionar eficientemente \textbf{múltiples almacenes físicos} distantes entre sí (Arequipa, Moquegua, Puno) y la falta de trazabilidad en productos sensibles como químicos de limpieza.

\begin{table}[h]
\centering
\begin{tabular}{|p{8cm}|p{8cm}|}
\hline
\textbf{Problema Detectado (Diagnóstico)} & \textbf{Solución Tecnológica (Odoo)} \\ 
\hline

\textbf{Incertidumbre de Stock:} Dificultad para conocer la disponibilidad de productos en tiempo real al momento de la venta. 
& 
\textbf{Módulo de Inventario Integrado:} Visualización de stock disponible en tiempo real con bloqueo automático de ventas sin existencias. 
\\ \hline

\textbf{Descontrol Multi-Sede:} Gestión manual o desconectada de los tres almacenes, dificultando los traslados internos. 
& 
\textbf{Gestión Multi-Almacén Nativa:} Configuración centralizada de sedes con flujo automatizado de reabastecimiento. 
\\ \hline

\textbf{Procesos Manuales en Despacho:} Emisión de guías de remisión y facturas desconectadas de la orden de venta. 
& 
\textbf{Facturación y Guías Electrónicas:} Emisión automática de comprobantes integrada con SUNAT desde la orden de venta. 
\\ \hline

\textbf{Falta de Trazabilidad:} Riesgo de vender productos vencidos por falta de control de lotes. 
& 
\textbf{Trazabilidad por Lotes:} Gestión de números de lote y fechas de caducidad para productos químicos. 
\\ \hline

\end{tabular}
\end{table}

\textbf{CM LINE E.I.R.L.} es una empresa comercializadora con presencia regional en el sur del Perú, especializada en la distribución de \textbf{materiales de construcción} (cementos, fierros), artículos de ferretería, productos de limpieza y \textbf{Equipos de Protección Personal} (EPP).

Su modelo de negocio es híbrido, combinando la \textbf{venta corporativa} (cotizaciones formales, volúmenes altos y gestión de crédito) con la \textbf{venta de mostrador} (retail), lo que exige una infraestructura tecnológica ágil que soporte tanto transacciones rápidas como flujos documentales complejos.

\subsection{Problemática identificada}
La principal problemática residía en la \textbf{fragmentación de la información}. Al no contar con un sistema centralizado, los traslados entre sedes se registraban con retraso o de forma inexacta, generando un desfase entre lo que el sistema contable indicaba y lo que realmente existía en percha. Esta falta de control afectaba no solo la rentabilidad, sino también la \textbf{confiabilidad ante el cliente}, dañando la reputación de la empresa por retrasos en las entregas.