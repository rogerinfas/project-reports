\section{Plan de Capacitación y Gestión del Cambio}

\subsection{Objetivo de la capacitación}
El objetivo principal del plan de capacitación fue mitigar la curva de aprendizaje del personal de ventas (jóvenes con perfil operativo), asegurando que perciban el sistema \textbf{Odoo} como una herramienta \textbf{"amigable"} y ágil. Se buscó garantizar la correcta adopción de los flujos de venta de mostrador y gestión multi-almacén, superando la resistencia inicial al cambio mediante el dominio práctico de la interfaz.

\subsection{Metodología de capacitación}
La capacitación se desarrolló mediante sesiones prácticas (talleres) utilizando la data real de la empresa (ej. cemento, detergentes), abordando los módulos de manera progresiva para no saturar al usuario:

\begin{itemize}
    \item \textbf{Inventario y Logística:} Uso de lectores de código de barras, gestión de traslados entre sedes (\textbf{Arequipa, Moquegua, Puno}) y recepción de mercadería.
    \item \textbf{Ventas y Facturación:} Creación de cotizaciones formales con términos y condiciones, y emisión de facturas/boletas integradas.
    \item \textbf{Punto de Venta (Mostrador):} Entrenamiento intensivo en la interfaz de \textbf{"Venta Rápida"} para despachos inmediatos sin cotización previa, simulando la agilidad de un sistema POS.
    \item \textbf{Trazabilidad y Compras:} Gestión de lotes y fechas de vencimiento para productos de limpieza y generación de órdenes de compra para reposición.
\end{itemize}

\subsection{Cronograma de capacitaciones}
Las capacitaciones se ejecutaron de manera paralela a los entregables (sprints). Se inició con la capacitación a los encargados de almacén tras la configuración de las sedes (\textbf{Sprint 1}) y continuó con la fuerza de ventas una vez habilitada la interfaz de facturación, permitiendo recibir retroalimentación inmediata para ajustar la \textbf{"usabilidad"} del sistema antes de la migración final.

\subsection{Evaluación de adopción del sistema}
La adopción del sistema se evalúa en función de la capacidad del personal para operar sin supervisión constante, evidenciándose:

\begin{itemize}
    \item Eliminación de consultas telefónicas para verificar stock en otras sedes.
    \item Uso correcto de la pistola de código de barras para el despacho de productos.
    \item Generación autónoma de \textbf{Guías de Remisión Remitente} desde el pedido de venta.
    \item Confianza del equipo comercial al bloquear ventas de productos sin stock real.
\end{itemize}
