\section{Plan de Capacitación y Gestión del Cambio}

\subsection{Objetivo de la gestión del cambio}
La estrategia de capacitación tuvo como propósito principal facilitar la transición tecnológica del personal administrativo y de ventas hacia el nuevo entorno digital. Se priorizó la reducción de la brecha de aprendizaje en los \textbf{perfiles operativos}, asegurando que el equipo perciba a \textbf{Odoo 19} como un ecosistema ágil y facilitador de tareas. El plan se centró en garantizar la adopción fluida de los procesos \textbf{multi-sede} y la estandarización de las operaciones de mostrador, superando la resistencia natural al cambio mediante el entrenamiento basado en escenarios reales.

\subsection{Metodología de transferencia de conocimientos}
La capacitación se ejecutó mediante talleres prácticos dirigidos, utilizando datos operativos reales de la empresa (gestión de materiales de construcción, detergentes y químicos). Se implementó un enfoque de aprendizaje progresivo estructurado en cuatro bloques fundamentales:

\begin{itemize}
    \item \textbf{Gestión Logística Integral:} Entrenamiento en el uso de dispositivos de lectura de códigos de barras para la recepción de mercadería y la ejecución de traslados internos entre los almacenes de \textbf{Arequipa, Moquegua y Puno}.
    \item \textbf{Ciclo Comercial y Facturación:} Capacitación en la elaboración de cotizaciones corporativas dinámicas (manejo de unidades de medida) y la emisión automática de comprobantes de pago electrónicos.
    \item \textbf{Punto de Venta de Alta Agilidad:} Entrenamiento intensivo en la interfaz de \textbf{Venta Rápida (PoS)} para el despacho inmediato de mercadería, simulando flujos de alta demanda en mostrador.
    \item \textbf{Control de Calidad y Reabastecimiento:} Instrucción en el registro de lotes, monitoreo de fechas de vencimiento y generación de \textbf{Órdenes de Compra} basadas en sugerencias de stock mínimo.
\end{itemize}

\subsection{Despliegue y Retroalimentación}
El cronograma de capacitaciones se integró con el avance de los \textbf{sprints} de configuración. Esto permitió que los encargados de almacén y la fuerza de ventas validaran la \textbf{"usabilidad"} del sistema en tiempo real, facilitando ajustes preventivos antes de la puesta en marcha definitiva.

\subsection{Resultados de la adopción tecnológica}
La efectividad del plan de capacitación se evidencia en la autonomía alcanzada por el personal, destacando los siguientes hitos operativos:
\begin{itemize}
    \item Eliminación total de la dependencia de consultas informales para verificar existencias globales.
    \item Dominio técnico en el uso de terminales de escaneo para el despacho preciso de productos.
    \item Capacidad autónoma para la generación de \textbf{Guías de Remisión Remitente}, cumpliendo con el flujo logístico-fiscal integrado.
    \item Incremento en la seguridad operativa del equipo comercial al trabajar bajo un sistema de bloqueo por falta de disponibilidad física.
\end{itemize}
