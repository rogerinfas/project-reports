\section{Introducción}

El presente informe de implementación documenta el proceso de adopción,
configuración y puesta en marcha del sistema ERP Odoo en la empresa
CM LINE E.I.R.L., con el objetivo de digitalizar y optimizar sus flujos
críticos de gestión logística, comercial y de facturación electrónica.

CM LINE es una empresa dedicada a la comercialización y distribución
de materiales de construcción, artículos de limpieza, ferretería y
equipos de protección personal (EPP). Con presencia en el mercado
regional a través de múltiples sedes (Arequipa, Moquegua y Puno),
la organización enfrentaba desafíos operativos significativos derivados
de la falta de visibilidad de inventarios en tiempo real y la
desconexión entre el área de ventas y almacén.

Esta situación generaba ventas sin respaldo de stock, quiebres no
planificados y sobrecostos operativos asociados a procesos manuales,
afectando la eficiencia del despacho y la satisfacción del cliente.

En este contexto, la implementación del sistema ERP Odoo se plantea
como una solución estratégica para integrar la gestión multi-almacén
de la empresa. El proyecto busca no solo la actualización tecnológica,
sino también asegurar la autonomía de los datos y establecer reglas
de negocio claras, tales como alertas de stock mínimo y bloqueo de
ventas sin disponibilidad real, garantizando eficiencia operativa
y trazabilidad de productos.

El presente documento describe de manera estructurada las fases del
proyecto, desde la configuración de los almacenes y la carga de datos
maestros, hasta la integración de la facturación electrónica con
guías de remisión. Asimismo, se detallan los resultados obtenidos en
la centralización de la información comercial y logística, evidenciando
una gestión escalable y adaptable a los distintos canales de venta
de la empresa.
