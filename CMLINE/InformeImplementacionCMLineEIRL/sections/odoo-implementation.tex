\section{Implementación del Sistema Odoo 19 ERP}

La implementación del sistema ERP \textbf{Odoo versión 19} en \textbf{CM LINE E.I.R.L.} se realizó bajo la modalidad \textbf{Odoo Online (SaaS)}, lo que permitió desplegar una solución centralizada en la nube capaz de interconectar en tiempo real las operaciones de las tres sedes físicas de la empresa (\textbf{Arequipa, Moquegua y Puno}) sin requerir infraestructura de servidores locales ni costos elevados de mantenimiento TI.

Esta modalidad garantiza la autonomía de los datos y la escalabilidad del sistema, permitiendo a la empresa gestionar múltiples almacenes sin incurrir en costos adicionales por la habilitación de nuevas sucursales, a diferencia de otras alternativas evaluadas en el diagnóstico inicial. Asimismo, asegura el acceso remoto seguro mediante credenciales únicas para cada perfil de usuario (\textbf{Ventas, Almacén, Administración}), facilitando la supervisión gerencial desde cualquier ubicación.

El sistema fue configurado estrictamente de acuerdo con los flujos operativos de \textbf{CM LINE}, priorizando tres pilares fundamentales:

\begin{enumerate}
    \item \textbf{Lógica de abastecimiento y traslados internos:} Centralización del inventario para evitar quiebres de stock.
    \item \textbf{Trazabilidad por lotes:} Control de calidad y fechas de vencimiento en productos de limpieza y químicos.
    \item \textbf{Dualidad comercial:} Integración de la venta corporativa formal y la venta directa de mostrador en una sola plataforma.
    \item \textbf{Facturación Electrónica e integración SUNAT:} Emisión automatizada de comprobantes de pago (Facturas, Boletas, Notas de Crédito) y Guías de Remisión Remitente, asegurando el cumplimiento tributario sin salir del sistema.
\end{enumerate}

La integración con SUNAT se configuró para que la validación de los documentos sea inmediata, eliminando la necesidad de procesos externos y garantizando que cada venta refleje su correspondiente obligación fiscal en tiempo real. Esta funcionalidad es crítica para \textbf{CM LINE}, ya que centraliza la legalidad de los traslados de mercadería entre provincias mediante las Guías de Remisión Electrónicas integradas.
