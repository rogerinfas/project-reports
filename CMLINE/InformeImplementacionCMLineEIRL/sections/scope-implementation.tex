\section{Alcance de la Implementación}

\subsection{Alcance temático}
El proyecto comprende la implementación y parametrización de los siguientes módulos del ERP \textbf{Odoo 19}, seleccionados estratégicamente para cubrir la operatividad logística y comercial de la empresa:

\begin{itemize}
    \item \textbf{Inventario (Multi-Almacén):} El módulo de Inventario de Odoo 19 gestiona de manera integral el control de existencias, permitiendo el registro de entradas, salidas y transferencias internas. En el caso de \textbf{CM LINE}, este módulo es la columna vertebral del sistema, ya que se configurará para administrar simultáneamente las tres sedes (Arequipa, Moquegua y Puno). Su implementación incluye la gestión de trazabilidad por lotes y fechas de vencimiento (crítico para productos de limpieza y químicos), así como la habilitación de campos para lectura de códigos de barra, garantizando la exactitud del stock y eliminando las \textbf{"ventas a ciegas"}.

    \item \textbf{Ventas:} El módulo de Ventas digitaliza el ciclo comercial completo, desde la creación de clientes hasta la confirmación del pedido. Para \textbf{CM LINE}, este módulo se adapta para gestionar múltiples unidades de medida (Unidad, Ciento, Millar) con listas de precios diferenciadas. Permite la generación de cotizaciones formales con términos y condiciones personalizados, el bloqueo de ventas si no hay stock disponible y el envío de documentos comerciales vía correo electrónico y WhatsApp, centralizando el historial del cliente.

    \item \textbf{Facturación Electrónica y Guías de Remisión:} Este módulo permite la emisión de comprobantes fiscales validados por SUNAT. En \textbf{CM LINE}, la implementación integra la emisión de Facturas, Boletas y Notas de Crédito directamente desde la orden de venta, eliminando la doble digitación. Además, se incluye la configuración explícita de Guías de Remisión Remitente para asegurar el traslado legal de la mercadería entre almacenes y hacia el cliente final, cumpliendo con la normativa vigente.

    \item \textbf{Punto de Venta / Venta Directa:} Módulo orientado a gestionar las ventas presenciales de manera ágil. En \textbf{CM LINE}, esta funcionalidad responde a la necesidad crítica de la \textbf{"venta de mostrador"}, donde el cliente requiere una atención inmediata (tipo retail/ferretería). Su objetivo es simplificar la interfaz para que los vendedores puedan despachar y emitir el comprobante rápidamente sin pasar por el flujo complejo de cotizaciones, mejorando la experiencia del cliente en tienda física.

    \item \textbf{Compras:} El módulo de Compras administra el proceso de adquisición de mercadería a proveedores. En \textbf{CM LINE}, este módulo se incorpora para gestionar la reposición de stock mediante Órdenes de Compra formales, permitiendo planificar el abastecimiento de los tres almacenes basándose en reportes de rotación de inventario y asegurando que los ingresos de mercadería actualicen automáticamente las existencias.
\end{itemize}

La implementación conjunta de los módulos de Inventario, Ventas, Facturación, Venta Directa y Compras permite a \textbf{CM LINE} contar con un ecosistema digital unificado que conecta sus almacenes en tiempo real con el área comercial, asegurando la trazabilidad de lotes, la agilidad en el despacho y el cumplimiento tributario.

\subsection{Alcance organizacional}
La implementación del ERP Odoo impacta transversalmente en la operación de la empresa, integrando sus tres sedes físicas (\textbf{Arequipa, Moquegua y Puno}) con la gestión central. El sistema está diseñado para cubrir las necesidades de las siguientes áreas clave:

\begin{itemize}
    \item \textbf{Gerencia General:} Acceso a reportes estratégicos para la toma de decisiones, supervisión de márgenes de ganancia y aprobación de compras para reposición de stock.
    \item \textbf{Logística y Almacén:} Gestión operativa de los tres almacenes, control de entradas/salidas, traslados internos entre sedes y trazabilidad de productos por lotes (limpieza y químicos).
    \item \textbf{Comercial (Ventas):} Dividida en dos frentes: fuerza de ventas corporativa (cotizaciones formales) y atención en mostrador (venta rápida), quienes requieren acceso a listas de precios y disponibilidad de stock en tiempo real.
    \item \textbf{Facturación y Caja:} Emisión de comprobantes electrónicos (facturas/boletas) y guías de remisión remitente integradas con SUNAT.
\end{itemize}

El sistema será utilizado por distintos perfiles de usuario, configurados con roles y permisos específicos:

\begin{itemize}
    \item \textbf{Usuario Administrador:} Control total de la configuración, acceso a costos y márgenes, y gestión de usuarios.
    \item \textbf{Usuarios de Ventas / Mostrador:} Perfil operativo diseñado con una interfaz simplificada (``amigable'') para vendedores jóvenes, enfocado en la agilidad de atención y verificación de stock sin acceso a configuración sensible.
    \item \textbf{Usuarios de Almacén:} Perfil enfocado en la recepción de mercadería, preparación de pedidos (picking) y uso de lectores de código de barras.
\end{itemize}

\subsection{Alcance temporal}
El proyecto se ejecuta de manera progresiva siguiendo un cronograma de seis fases (sprints), iniciando con la configuración logística de las tres sedes y el control de inventarios. Posteriormente, se integran los flujos de ventas corporativas y facturación electrónica, y culminará con la habilitación de la venta rápida de mostrador y la migración masiva de datos reales para su puesta en operación completa.

\subsection{Cronograma y Estado de Implementación}

A continuación, se detalla el estado actual de las fases del proyecto en \textbf{CM LINE}:

\begin{table}[h]
\centering
\renewcommand{\arraystretch}{1.5}
\begin{tabular}{|l|p{8cm}|p{5cm}|}
\hline
\textbf{Fase} & \textbf{Módulo / Actividad Clave} & \textbf{Estado Actual en CM LINE} \\ \hline

Fase 1 & Inventario y Logística (Multi-Almacén): Configuración de las 3 sedes (Arequipa, Moquegua, Puno), categorías y unidades de medida. Traslados internos probados. & \textbf{Completado} \\ \hline

Fase 2 & Ventas Corporativas (Cotizaciones): Flujo de cotización formal, PDF con términos y condiciones, y envío por correo electrónico. & \textbf{Completado} \\ \hline

Fase 3 & Facturación Electrónica y Guías: Emisión de facturas y configuración de Guías de Remisión Remitente. & \textbf{En Espera} (Habilitación de usuario secundario SOL por parte del cliente) \\ \hline

Fase 4 & Venta Directa (Tipo POS): Interfaz rápida para venta de mostrador (sin cotización previa). Es el punto crítico que exige el cliente ("Estilo Tumisoft"). & \textbf{Completado} \\ \hline

Fase 5 & Compras y Abastecimiento: Generación de Órdenes de Compra para reposición de stock. & \textbf{Completado} \\ \hline

Fase 6 & Trazabilidad y Lotes: Configuración de fechas de vencimiento y números de lote para productos de limpieza/químicos. & \textbf{Completado} \\ \hline

Fase 7 & Migración y Despliegue Final: Carga masiva de clientes y stock real mediante plantillas de Excel. & \textbf{Completado} \\ \hline

\end{tabular}
\end{table}
