\section{Conclusiones}

\begin{itemize}
    \item La implementación del ecosistema \textbf{Odoo 19} ha permitido a \textbf{CM LINE E.I.R.L.} consolidar la operatividad de sus tres almacenes (\textbf{Arequipa, Moquegua y Puno}), erradicando los silos de información y estableciendo una \textbf{visibilidad de stock en tiempo real} indispensable para la toma de decisiones gerenciales.
    \item Se ha resuelto de manera definitiva la problemática de las \textbf{"ventas a ciegas"}, gracias a la instauración de una lógica de inventario estricta que impide la facturación de existencias no verificadas, optimizando así los márgenes operativos y la satisfacción del cliente.
    \item El sistema ha estandarizado la \textbf{dualidad comercial} de la empresa, integrando con éxito los procesos de alta complejidad corporativa (cotizaciones masivas por unidades de medida) con la agilidad requerida en el despacho de mostrador mediante el Punto de Venta.
    \item Se ha establecido un nuevo estándar de \textbf{control de calidad} para los productos sensibles de limpieza y químicos a través de la trazabilidad por lotes, garantizando la rotación adecuada de productos según su fecha de vencimiento y reduciendo las pérdidas por merma.
    \item La unificación del flujo logístico-tributario, mediante la vinculación directa de la \textbf{Facturación Electrónica} y las \textbf{Guías de Remisión} con las órdenes de venta, ha reducido drásticamente la carga de digitación manual y garantiza el cumplimiento normativo ante \textbf{SUNAT}.
\end{itemize}
