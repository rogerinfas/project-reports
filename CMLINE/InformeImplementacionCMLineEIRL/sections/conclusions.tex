\section{Conclusiones}

\begin{itemize}
    \item La implementación del ERP \textbf{Odoo 19} permitió a \textbf{CM LINE E.I.R.L.} superar la gestión fragmentada de sus tres almacenes, logrando una visibilidad integral del stock en \textbf{Arequipa, Moquegua y Puno} desde una única plataforma en la nube.
    \item Se logró resolver el problema crítico de las \textbf{"ventas a ciegas"}, estableciendo reglas de negocio que impiden facturar productos sin disponibilidad física, lo que elimina costos operativos por errores de despacho.
    \item El sistema integró exitosamente la \textbf{dualidad comercial} de la empresa: soporta procesos complejos de cotización corporativa (con diferentes unidades de medida) y, simultáneamente, procesos ágiles de venta directa en mostrador.
    \item Se elevó el estándar de control de calidad mediante la incorporación de \textbf{trazabilidad por lotes}, permitiendo gestionar fechas de vencimiento en productos de limpieza y químicos, reduciendo el riesgo de mermas.
    \item La gestión administrativa se automatizó al vincular la \textbf{Facturación Electrónica} y las \textbf{Guías de Remisión} directamente a la orden de venta, reduciendo la carga operativa y asegurando el cumplimiento tributario con \textbf{SUNAT}.
\end{itemize}
